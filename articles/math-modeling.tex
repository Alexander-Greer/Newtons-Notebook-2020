\nnarticleheader{Math Modeling Competition}{Mickey Fairorth '19, Nikhil Chakraborty '19, Alexander Greer '20}

This past March, three Haverford students were given 14 straight hours and the task of using mathematics to model drug use trends in the United States. These three students, Nikhil Chakraborty, Alexander Greer, and Mickey Fairorth, participated in a nationwide challenge called the Mathworks Math Modeling (M3) Challenge. This challenge is open to all high school juniors and seniors and aims to encourage students to work as a team to tackle real world problems. Topics in years past have ranged from climate change and its effects on sea levels to food insecurity around the world. This year’s challenge required the students to analyze an issue that plagues the nation: substance abuse and addiction.

The problem contained three separate parts. The first question focused on high school use of nicotine, particularly through e-cigarettes or vapes. The second question asked the students to analyze factors that lead to substance abuse and addiction. These factors could include income, race, mental illness, and social behavior. The third and final question highlighted the damaging and lasting impact of substance abuse. Specifically, the three students were asked to make a model that included both financial and non-financial impacts of substance abuse and then use the model to determine which drugs have the most detrimental impacts.

At the end of the 14 hours, the students had created a nearly 20 page report that included an executive summary, graphs and charts, and a Python script that was used to fit trends. Below is the executive summary and their answer to the first question.

\section{Test}


\setcounter{section}{0}