\nnarticleheader{Cryptography}{Alexander Greer '20}
With internet technology becoming ever more omnipresent, a major concern expressed in industry and by individuals alike has been cybersecurity. How can we make sure that the data we transmit and receive online can only be seen by certain people?

Speaking generally, the data transmitted between your devices and various others around the world are strings of bits – ones and zeros in a specific order that can be converted into information. The way this “binary” information is converted has been standardized, which is great for the globalization of communication, as everybody can translate each other’s messages. It also means that anybody who gets access to your binary data can read it, even if you don’t want them to. To prevent this, systems of “data-encryption” were developed, leading to the expansion of the field of cryptography.

Data encryption is basically the digital version of a secret code, the key to which is only given to certain people. This usually involves taking an input (a file, a message, any piece of transcribable information) and transforming it into a new piece of information using this special code.The way the input is transformed differentiates secure data encryption schemes from insecure ones.

Imagine you and your friends want to keep a ledger of all of each other’s transactions. As a way to verify each transaction, you decide that each party must sign the ledger. This seems pretty secure, but what is stopping one of your friends from copying everybody’s signature and forging transactions? In the digital world, this would be much easier, as a “signature” would just be a string of bits that could be copied by a computer. To solve this, you and your friends invent a scheme called “asymmetric” or “public key” encryption, which relies on  private and public keys. Everybody has access to everyone else’s public keys, but each individual keeps their private key a secret. Whenever you sign the ledger, you use your private key to generate a signature unique to that transaction, one that cannot be copied anywhere else. Your corresponding public key can be used for verification by combining it with the transaction and the generated signature. 


\begin{figure}[H]
    \centering \includegraphics*[scale=.25]{assets/cryptography.png}
\end{figure}


In the real world, this concept is not just limited to transactions, but is used for most forms of digital communication. One function used to generate your so-called digital “signature” is called a Hash Function. The SHA-256 Hash Function, developed by the NSA in 2001, is the one used by most web browsers to protect your personal information. As the name suggests, it encrypts blocks of data into 256-bit chunks (strings of 256 ones or zeros). The SHA function is what is called a “one-way compression” function. This means that generating a signature from an input results in an output that cannot feasibly be reverse-engineered back into the original message. Part of what makes SHA so difficult to reverse-engineer is that SHA functions work using bitwise functions. These are functions that, instead of working with the original message, convert the message to bits and manipulate the ones and zeros individually, resulting in a high level of encryption that does not fall prey to common decryption schemes like counting letter frequency.

Another simple yet effective technique used to prevent reverse decryption is the modulus function, represented by the percentage character “\%.” The modulus is a mathematical operation (just like plus and minus) that takes an input of two numbers and outputs their remainder when divided. Here are a few examples:

\[19 \% 7 = 5\]

\[15 \% 4 = 3\]

\[48 \% 7 = 6\]

\[18 \% 9 = 0\]

Because there are an infinite number of inputs that could result in the same output (e.g. 15 \% 4 = 3 and 37 \% 17 = 3), it is impossible to directly reverse-engineer outputs, thus making the function extremely useful in effective data encryption.

Regardless of exactly how any of these functions work, just be thankful that the smart people who developed them had the interest of your digital privacy in mind. 


