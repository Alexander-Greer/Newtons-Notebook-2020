\nnarticleheader{Foundations of Mathematics: Part I}{Dr. Mark Gottlieb, Haverford Faculty}
\noindent
\begin{quotation}
\textit{The science of pure mathematics ... may claim to be the most original creation of the human spirit.}
\begin{flushright}
Alfred North Whitehead
\end{flushright}
\end{quotation}
\noindent
\textbf{Introduction}

     Although we are all familiar with basic mathematical ideas and operations, most of us would be at a loss if we were asked to define mathematics, to say what mathematics "really is".  We know, for example, that mathematics studies numbers, shapes, and patterns.  But what is a number?  Although numbers seem to be important in nearly all practical situations, a number does not seem to be a material object.  For example, unlike physical objects, numbers do not seem to undergo change.  Physical things age and, eventually, fall apart.  Numbers do not.  Whatever properties a number has, it seems that it has always had them and always will.  For example, the number 7 is prime because it has no whole number factors other than 1 and 7.  That makes it different from 6, for example, because 6 = 2 x 3.  It seems clear that 7 will always be prime and that 6 never will be. 
      
     
     Numbers, unlike physical objects, are not things we perceive by means of the senses of  sight, hearing, or touch.  In a similar way the triangles, circles, and squares that we study in geometry, if we think about it carefully, are not perceived by the senses, either.  In this case we feel more confident in asserting that these objects belong to the "real world" of space.  For example, we may feel that we can draw a triangle on a blackboard or construct a circle on a piece of paper using a compass.  If we think carefully about this, however, we realize that, no matter how accurately we draw our picture, it will, at least in some small respect, remain only an imperfect copy of what we imagined.  In point of fact, no one can ever draw a triangle, circle, or square.  Nevertheless, like numbers, geometrical objects seem to have unchanging properties.  For example, the sum of the measure of the angles of a triangle is always $180^{\circ}$.\footnote{Interestingly, we might be inclined to doubt this were we restricted to making measurements with a protractor, since we would almost certainly measure the sum to be different from exactly $180^{\circ}$. }
     
     We can now see that neither numbers nor geometrical figures are material objects, but what about patterns and relations?  Like numbers and shapes, patterns and relations seem to be very real features of the world.  And, just like numbers, they do not seem to be material objects.  What is going on here?  Mathematics seems to be the study of certain kinds of things that, while very much real, do not belong to the physical world.  At first, it may seem rather puzzling that there appear to be real things that are not physical, but the fact that we can acquire mathematical knowledge challenges our belief that all real things are physical.  It may surprise you that mathematicians themselves are deeply puzzled by this problem, as well.  

\noindent
\textbf{Realism}

     Until the beginning of the nineteenth century most mathematicians believed that they were studying non-physical objects that were unchanging and eternal.  They believed that, somehow, the human mind could "grasp" mathematical truths by directly "seeing" them, just as we "see", for example, that 7 + 5 = 12.  The view that mathematics is made up of eternal truths about unchanging objects is an ancient idea.  It goes back to the sixth century B.C. and is associated with the teachings of the Greek religious brotherhood known as the Pythagoreans.  According to these men, their founder, Pythagoras, had taught that the real world is not the way it seems to the senses, but is actually governed by invisible mathematical relationships.  The Pythagoreans seem to have made the discovery that the basic intervals of the Western musical scales, for example, the octave or the fifth, are the result of dividing a stretched string into whole number ratios.  Cutting the string in half, for example, results in raising the pitch by one octave.  Cutting it in a ratio of 2:3 results in raising it by the major fifth (e.g., from middle C to the G directly above it).  These remarkable discoveries must have been awe-inspiring to the men who made them, for they showed that the universe obeys exact mathematical relationships.  This inspired them to hold mathematics in the highest esteem as the very basis of the universe.  Interestingly, the Pythagoreans later experienced profound dismay when it was discovered that there are numbers such as $\sqrt{2}$   that are not equal to any fraction, however large the denominator.
     
     Many scholars today regard Plato as the greatest and most influential of the ancient Greek philosophers.  As a young man he became acquainted with the teachings of earlier Greek thinkers, like the Pythagoreans, who believed that mathematics was the study of unchanging objects. Throughout his many writings, he expressed the view, quite similar to that of the Pythagoreans, that the world as it appears to the senses is not completely real.  Instead, beyond the world of the senses lies the world of Ideas or Forms which are perfect and unchanging.  For example, Plato argued that, while any particular tree was always changing and gradually decaying, the concept or Form, "Tree", always remained the same.  Scholars often refer to Plato's view as his "Theory of Forms", and numbers, shapes, and other mathematical objects are among the Forms.  Mathematicians who share Plato's view are, naturally, known as Platonists or mathematical realists.
     
    Realism appeals to mathematicians for a number of reasons.  It seems to give mathematics a special significance and dignity among the sciences.  First, as we have mentioned, mathematics seems to be special mainly because it is the only subject that studies unchanging objects.  Platonists believe these objects have always existed and will always exist.  Furthermore, because mathematical facts come from the properties of  these objects, no human being can alter them.  Mathematical truth, according to the Platonist, is objective  and independent of human minds.  Mathematical research, just like research in the sciences, is the attempt to discover, not to create, new mathematical facts and ideas.  Finally, mathematical ideas seem to be totally clear and distinct, unlike the concepts used in other subjects, like biology or history.  Realists explain this clarity by pointing to the fact that mathematical objects are not made up freely by the human mind, but have a perfect form in and of themselves.  Although they may not be quite clear about it in their own minds, it is probably fair to say that most mathematicians engaged in theoretical research today are realists, at least in spirit.  Interestingly, however, very few eminent mathematicians in recent times have actually called themselves realists.  Notable exceptions include Kurt Gödel, perhaps the greatest logician of the twentieth century, and the French mathematician René Thom, who helped to establish a branch of mathematics called \emph{chaos}.  
    
    If Platonism expresses so many common beliefs about mathematics, why have so few mathematicians been willing to call themselves Platonists?  Platonism is an appealing viewpoint for mathematicians, but it is very difficult to justify the idea that mathematical objects have no connection to the material world.  For example, if mathematical objects exist in a non-physical world, how do our brains, which are obviously physical, ever come into contact with them?  How does the number 7 or a right triangle get inside my brain so I can think about it?   Perhaps in my brain there are only numerals, or symbols for numbers.  But, then, how do these numerals connect to numbers?  This is a mystery.  It seems that numbers must be somehow connected with the physical world if we are to be able to understand mathematics (or, for that matter, to teach it to others).\footnote{On this point see R. Hersch, \textit{What is Mathematics, Really?}.}  In science, particularly in physics, mathematical equations are used to represent events and relationships in the material world, what physicists call "laws of nature."  If mathematical objects do not belong to the physical world, it is very difficult to see how or why physical laws should be mathematical.  Along these lines, the physicist Eugen Wigner once remarked that mathematics was "unreasonably effective" in physics.  

\noindent
\textbf{Logicism}

     There is one further puzzle that Platonism creates but doesn't solve.  If mathematical objects exist in a non-physical place, and if we are able to recognize mathematical truths by directly "seeing" them, why do mathematicians find it necessary to go through careful reasoning or proof  in order to be sure they are reaching correct conclusions?  All of modern mathematics is based upon the idea of  justifying our conclusions by going back to a set of very basic assumptions, called axioms or postulates.  Clearly, if we are to prove something, we have to start somewhere.  If we had no background knowledge whatsoever, we could never reach any conclusions.
     
     For example, supposing we know that all birds are warm-blooded and that  penguins are birds, we can prove that penguins are warm-blooded.  We can depict our reasoning process as follows:

\begin{center}
Assumption 1: All birds are warm-blooded. Assumption 2: All penguins are birds.

Conclusion: All penguins are warm-blooded.
\end{center}  
An example of a very simple mathematical argument is the following:
\begin{center}
Assumption 1: A = B. Assumption 2: B = C.

Conclusion: A = C.
\end{center}

Mathematicians call this pattern of reasoning the \emph{Transitive Property of Equality}.
In both of the cases above, the conclusion follows necessarily from the assumptions.  It is not possible for the assumptions to be true and the conclusion to be false.  It is time to introduce our first formal definitions.
\begin{dfn}
A \textbf{deductive argument} is any set of statements in which one statement, the conclusion, is marked off from the other statements.
\end{dfn}
\begin{dfn}
A \textbf{proof} is a collection of statements placed in an order such that the last statement, called the conclusion, follows from the truth of the other statements.  In mathematics, the conclusion is called a \textbf{theorem}.   Note that a deductive argument may or may not be a proof, since the conclusion may or may not follow logically from the other statements.
\end{dfn}
\begin{dfn}
A(n) \textbf{hypothesis} of  a deductive argument is an assumption or starting point of the argument.
\end{dfn}

The idea of building up mathematical knowledge by a chain of deductive arguments starting from axioms or postulates goes back to the ancient Greeks, who excelled in both logic and geometry.  The most important example of this way of doing mathematics is contained in the work of  Euclid, who brought together much of was known about geometry in a single book, the \emph{Elements}.  In this book Euclid identified a basic set of geometrical axioms (unproven assumptions)  from which every other result could be proven.  This was an extraordinary achievement and has exerted a tremendous influence upon all of mathematics, science, and philosophy in the Western world.  Interestingly, until the twentieth century, mathematicians did not succeed in building up arithmetic and algebra, which study numbers and number operations, from a set of simple axioms.  Today, there are still unresolved problems about how to accomplish this goal correctly.

     Some mathematicians, impressed by the power of logic and proof, have tried to solve the problems created by Platonism by arguing that mathematics is really just a collection of deductive arguments, starting from a set of basic, unprovable assumptions.  In other words, according to these mathematicians, mathematics is the same thing as careful reasoning or logic.  This viewpoint is called logicism.  Its most famous proponent  was the German philosopher Gottlob Frege, one of the creators of modern logic, which expresses deductive arguments using special symbols.  During the twentieth century, many mathematicians, wary of becoming Platonists, defended the position that mathematics is really logic, and many books about what is called mathematical logic were written.  Eminent philosophers, such as Bertrand Russell, W.V. Quine, and Rudolph Carnap, spent large parts of their careers developing systems of mathematical logic.  Despite these efforts, no one has ever been able to show how to express all of mathematics as pure logic.   It turns out that it is always necessary to bring in ideas that are not actually logical, but practical.
     
     We have to make sure, for example, that we don't end up in a muddle by talking about mathematical objects that can't really exist.  One of the main goals of the logicians mentioned above was to show that all the objects we work with in mathematics--numbers, spatial relationships, patterns--can be expressed in their most basic form as sets, or collections.  The study of sets was begun by the German mathematician Georg Cantor, who was especially interested in infinite sets.  Later in this book we will discuss how Cantor used the theory of sets to reach remarkable conclusions about infinite collections.  Let's make a few definitions pertaining to sets.
     
\begin{dfn}
A \emph{set} is any collection.  The members or \emph{elements} are the things that belong to the set.  We use the symbol $\in$ to indicate that a member belongs to a set.
\end{dfn}

We denote a set by using a variable symbol, usually a capital letter, and we represent that set by enclosing its elements within brackets.

\begin{exm}{1}
The set $\Re$ of real numbers can be represented by$\lbrace 1, 2.56, \pi, e, ...\rbrace$.  Here, the ellipsis ... means that the set has infinitely many members.
\end{exm}

\begin{exm}{2}
The set of all birds can be represented as B=$\lbrace$ my pet pigeon, the bald eagle in the zoo, ..., the University of Iowa mascot $\rbrace$.  Here the ellipsis replaces all the birds that are not written down (it would be a long list!).  Notice that a set does not have to contain numbers; it can have any members whatsoever.  This is a finite set.
\end{exm}

Bertrand Russell, one of the major proponents of logicism, recognized that there are sets that we can describe in words that simply make no sense.  Here is Russell's idea:  Some sets have the interesting property that they are \emph{members of themselves}.  For example, the set of \underline{all} sets is a set, so it is a member of itself.  As another example, consider the set T of all sets that contain more than 10 members.  This is an infinite set, and we can represent it as follows.
\begin{center}
$T=\lbrace\lbrace1,2,...11\rbrace,\lbrace1,2,...12\rbrace,...\rbrace$
\end{center}
 

Clearly T has more than 10 members, so T belongs to itself.  In our notation we write $T\in T$.  T is a set that contains itself as a member!  Obviously, this is not true of most sets.  For example, the set B of all birds above does not contain itself as a member; only birds belong to B, not sets.\\

     Now suppose we think about the set of all sets that are not members of themselves, like B.  Let's call these sets \emph{standard} sets\footnote{I borrow this term from S. Körner, \textit{Philosophy of Mathematics}, p. 45.} and denote the collection of all standard sets by S.  We can represent S as follows.
     \begin{center}
     S = $\{ B, \Re, \{1,2,3\},...\}$
     \end{center}

Here is Russell's question: Is S a standard set?  That is, is S a member of itself?  Well, if $S\in S$ , then S would seem to be standard, since it belongs to the set of standard sets, \underline{but} S would also seem to be non-standard, since it belongs to itself.  On the other hand, if $S\notin S$ , then S would seem to be standard, so $S\in S$  !  It looks like the following is true:

\begin{center}
$S\in S$ if and only if  $S\notin S$
\end{center}

It seems that S can neither belong to itself nor not belong to itself. What's going on here?

     This puzzle is known as \emph{Russell's Paradox}.  A logical paradox is a proof that has mutually contradictory conclusions.  Another famous logical paradox, which comes from the ancient Greeks, is called the \emph{Epimenides (or Liar) Paradox}.  Suppose a man says, "I am lying".  Is he telling the truth?  If he is, then he is lying, so he isn't telling the truth.  Conversely, if he isn't telling the truth, then he is telling the truth, since he is asserting that he is a liar.  His statement is paradoxical.
     
    Remember that the goal of the logicists was to show that mathematics is really just pure logic.  But there is nothing wrong with our reasoning when we reach a paradox.  In other words, logical thinking can lead to paradoxes.  On the other hand, we never seem to reach any paradoxes when we work with numbers, shapes, or patterns.  There are paradoxes in logic, but none in mathematics, so mathematics doesn't seem to be identical to logic.  In fact, the only way to avoid getting logical paradoxes is to introduce a set of rules that forbids us from talking about sets like S.  But these rules aren't logical, just conventional, like the rules of chess.  They are practical guidelines that keep the game going.  If these are really the basic rules of mathematics, then mathematics isn't logic.
    
     There is a further problem.  As mentioned earlier, one of the main goals of  logicians has been to put mathematics on a completely solid foundation by expressing all mathematical concepts in terms of sets. At one time, many mathematicians believed that the properties of sets, which we will study in detail later,  were so simple and obvious that they would make an ideal basis for expressing any mathematical idea whatsoever.  For example, we can express the number 2 as the set of all sets having exactly two members.  In geometry, we can express a circle as the set of all points lying at the same distance from a fixed point (the center). Russell's Paradox seems to show that sets aren't as simple as they seemed.  It appears that numbers, shapes, and patterns might be simpler, after all.  It looks like doing mathematics isn't the same thing as thinking logically.  
\newline\newline\newline\newline\newline
\begin{center}
	\textit{Part II of Dr. Gottlieb's "Foundations of Mathematics"\\ 
	(with sections on \textbf{Formalism} and \textbf{Constructivism})\\
	will be published next year in Issue V of} \\
	Newton's Notebook.
\end{center}