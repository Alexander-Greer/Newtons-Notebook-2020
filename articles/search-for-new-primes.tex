\nnarticleheader{The Search for New Primes}{Safa Obuz '21}

The study of prime numbers, a specific yet important branch of study in number theory, continues as a researched topic dating back from 2000 years ago and holds itself in increasing importance during the last century and into the future. Simply put, all of the modern world’s cryptography is based in our collective understanding of primes and their behavior. 

\subsubsection*{What is a Prime?}

A prime number is a natural number larger than 1 which cannot be expressed as the product of two smaller natural numbers. 
\[2, 3, 5, 7, 11, 13, 17, 19, 23, 29, 31, 37, 41, 43, 47, 53, 59 ,61, 67, 71, 73, 79, \ldots \]
They are effectively the analog of “atomic elements” of natural number multiplication. This comes from the one of the first theorems of primes.

\textbf{Fundamental theorem of arithmetic:}  (Euclid, \(\approx 300\) BCE) Every natural number larger than 1 can be expressed as a product of one or more primes. This product is unique up to rearrangement.

Other than the trivial rearrangement of the multiplication of primes, for instance, 50 can be \(2 * 5 * 5\) (or \(5 * 5 * 2\)), there is only 1 way to express 50 in terms of primes. The reason why 1 is not a prime is based on this theorem; if 1 was a prime, although it itself cannot be split into smaller natural numbers, the fundamental theorem of arithmetic would not be true. After this theorem came ideas about the amount of primes, and the conclusion that there are lots of them. The ancient Greeks studied primes meticulously, and they proved:

\textbf{Euclid’s theorem:} (Euclid, \(\approx 300\) BCE) There are infinitely many prime numbers.

The proof behind this concept is very simple and intuitive, and utilizes proof by contradiction, \textit{reductio ad absurdam}. Essentially, we assume a state of falsehood, and then prove that said falsehood is nonexistent. Indirectly, that proves a theorem. 

\begin{enumerate}
    \item Suppose there are only a finite amount of prime numbers, \(p_1, p_2 \ldots p_n\) (e.g.\ suppose 2, 3, 5 were the only primes).
    \item Multiply all the primes together and add (or subtract) 1: \(p_1, p_2 \ldots p_n \pm 1\) (e.g. \(P = 2 * 3 * 5 \pm 1 = 29, 31\)).
    \item As a result, \(P\) is a natural number larger than 1, but is not divisible by any of the prime numbers.
    \item This contradicts the fundamental theorem of arithmetic. Thus, there are infinitely many primes.
\end{enumerate}

Short and elegant, this theorem explains that there are infinitely many primes, but it doesn’t explain their location. There are more direct proofs that hunt for their location more explicitly, but none are as short or elegant. While this proves there are infinitely many primes, as of writing this, the largest prime we know, a Mersenne prime, is \(2^{82,589,933} - 1\); a number with 24,862,048 digits is, nevertheless, finite. Knowing all this, we can make a jump to the future to see the importance of primes, and see why the search for new primes and understanding about them is so valuable.

The whole business of showing the behavior of primes is not a purely academic concern, and these days, internet and banking security underlies mathematicians’ beliefs of the randomness of primes.  

\subsubsection*{Diffie-Hellman key exchange}

The real-world application is the Diffie-Hellman key exchange (1976), which is a secure way to allow two strangers (typically named Alice and Bob) to share a secret, even when their communication is completely open to eavesdroppers. This algorithm, combined with similar algorithms like the RSA, is often used in internet security protocols. 

An analogy to how it works is if Alice is sending secret message g by physical mail to Bob, and she knows someone is reading both incoming and outgoing mail. She has no other means of communication with Bob. First, she places her own lock on the outgoing mail, to which only she has the key. Then Bob places his own lock on the incoming mail and returns it back to Alice. This allows Alice to unlock her own lock with there still being a lock on the mail. Then she returns it to Bob, ensuring that only Bob can read the mail.  

The (simplified) Diffie-Hellman protocol behaves as follows:

\begin{enumerate}
    \item Alice and Bob agree (over the insecure network) on a large prime \(p\).
    \item Alice picks a key \(a\), which is a prime, “locks” \(g\) by computing \(g^a \pmod{p} \), and sends \(g^a \pmod{p} \) to Bob.
    \item Bob picks key \(b\), “double locks” \(g^a \pmod{p}\) by computing \({(g^a)}^b = g^{ab} \pmod{p} \), and sends \(g^{ab} \pmod{p} \) back to Alice.
    \item Alice takes the \(a\)th root of \(g^{ab} \pmod{p} \) to create \(g^b \pmod{p} \), to send back to Bob.
    \item Bob takes the \(b\)th root of \(g^b \pmod{p} \) to create \(g^b \pmod{p}\), to recover \(g \pmod{p}\).
\end{enumerate} 

This works because:
\[{(g^a \pmod{p})}^b \pmod{p} = g^{ab} \pmod{p}\]
\[{(g^b \pmod{p})}^a \pmod{p} = g^{ba} \pmod{p}\]

Even though the algorithm appears secure, we have no explicit proof. In fact, this issue is related to a \$1 million prize problem issued by the Clay Mathematics Institute: \(P \neq NP\). Simply put this equation asks: Is it possible for every solved problem whose answer is able to be quickly checked by a computer, to also be quickly solved? If this is true, then it is equally easy to decipher which keys Alice and Bob used and refute many other assumptions we formulated in the greater context of math.

The study of primes intertwines with other areas in mathematics to help us model equations to predict their location. Using the \textbf{Prime Number Theorem} (Hadamard \& Poussin, 1896), we can predict their distribution in relation to number of primes less than N. Extremely complex in nature yet visually simple, the equation
\[
\lim_{x\to\infty} \frac{\pi(x)}{[\frac{x}{\log{x}}]} = 1    
\]
builds on the teachings of Bernhard Riemann, and is accurate to -1.8\% at the \(10^3\) \(x\) or \(N\). This equation is generally regarded as the greatest single achievement in number theory, and with primes’ locations increasing relevance and human mathematical curiosity ever craving for answers, the Prime Number Theorem provides immense value. 

Every day new primes are being discovered, and more and more research is poured into this topic. The truly interesting portion of the whole endeavour lies in that humans have known about primes for centuries, yet we still know so little about them. 
