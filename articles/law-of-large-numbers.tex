\nnarticleheader{The Law of Large Numbers}{Chris Tsetsekos '20}

In probability theory, the Law of Large Numbers (LLN) is a theorem that describes the result of performing the same experiment a large number of times. According to the law, the average of the results obtained from a large number of trials should be close to the expected value, and will tend to become closer as more trials are performed.

The LLN is important because it guarantees stable long-term results for the averages of some random events. For example, while a casino may lose money on some spins of the roulette wheel, it will have predictable earnings over a large number of spins. Any winning streak by a player will eventually be overcome by the parameters of the game. It is important to remember that the law only applies (as the name indicates) when a large number of observations is considered. There is no principle that a small number of observations will coincide with the expected value or that a streak of one value will immediately be ``balanced'' by the others, which are ideas collectively referred to as the ``Gambler’s Fallacy.''

This past spring, I won 197 faceoffs out of 324 attempts for The Haverford School lacrosse team. This translates to a 60.8\% success rate. This past summer, I went 339-for-552. I took 228 more faceoffs in the summer than the spring; however, my success rate remained nearly identical: 61.4\%. This small difference – eight tenths of a percent – is a good example of the Law of Large Numbers: no matter how many faceoff attempts I take, it is highly likely that my success rate will be the same. 
