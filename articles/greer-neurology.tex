\nnarticleheader{The Root of Consciousness and Mapping the Brain}{Alexander Greer, Haverford '20}
\noindent
\textbf{Introduction}

A lot of things are happening in your head right now as you read the words on this page. You are moving your eyes around to follow each line. Your eyes receive light reflected by the page which is interpreted as vision. One part of your brain processes that vision to recognize text. Another, as it recognizes letters, words, and phrases, may sound them out as if being spoken. Another still will interpret the meaning of the text and associate it with your memories, perceptions, and senses \cite{urry}. The goal of scientists and neurologists, for hundreds of years, has been to understand how we are able to combine all of these distinct regions so effectively to perceive and interpret the world around us with unmatched complexity. We have a basic understanding of what each portion of the brain does, yet a more complex model of how they work together proves to be our largest challenge \cite{turk}. Understanding how the brain does what it does will pave the way for treatments to psychological disorders, developments into modified and artificial intelligence, and answers to the question of what it means to be human.

\noindent
\textbf{Part I – How do brain cells communicate?}

You may remember from high school biology that the cells in your brain, called neurons, do a lot of connecting with one another. Neurons have axons and dendrites, which send and receive signals, respectively, between other neurons. We have a pretty good understanding of how neurons interact on the smallest level from cell studies using brain tissue: the movement of charged molecules called ions across the surface of these cells propagates an electrical signal that can be received and transmitted by other cells \cite{urry}. In this sense, the connections between neurons in your brain are like a computer sending bits of information – one/zero, on/off, yes/no – through wires and circuits. 

This kind of binary thinking has pointed to computer science as a means of understanding such complex networks. In fact, the branch of computer science that focuses on artificial intelligence closely parallels the efforts of neuroscientists. Researchers are creating neural networks (NNs), essentially digital simulations of the connections seen between neurons in the brain, to try to predict complex nervous system behavior. These models consist of nodes (analogous to signal-receiving dendrites) and links (analogous to signal-transmitting axons) formed together in complex, web-like patterns. Each node has a set of instructions for how to pass its inputs onto other links, much like how the physical arrangement of neurons in the brain can influence the way signals travel. Also notable were “weights” given to each link which determined the strength of the signal sent from one node to the next. Since the electrical signals transmitted across neurons are always of the same magnitude, this system was instead meant to replicate the likelihood and frequency of signal transduction based on the node’s input \cite{rubinov}. This parallels a neurological concept whose implications are still being researched: the effects that different neurotransmitters, chemicals that trigger or halt the sending of electrical signals, have in the brain.

The places where neurons meet in your brain are called synapses. They are characterized by the release of these neurotransmitter molecules for cell communication. Neurotransmitters can have excitatory or inhibitory effects on the receiving cell, either moving it towards or away from sending transmitting own electrical signals. Chemicals like dopamine and epinephrine, which control things like mood and attention, are excitatory, while others such as serotonin, which contributes to sleep, are inhibitory. Obviously these molecules must influence more than just micro-scale, neuron-to-neuron interactions in order to enact their associated functions as described above. X-ray studies have revealed that individual neurotransmitters can play larger roles in specific regions of the brain. Molecules can be mapped onto a sort of brain mosaic that shows how each is distributed during brain function. The concentrations of neurotransmitters are currently informing our understanding of connections between the functional regions of the brain.

\noindent
\textbf{Part II – How is brain tissue organized?}

A 2012 study aimed to figure out how the many different regions of the brain work together. These researchers, at the University of California, Berkeley, used a common uniting factor to associates multiple regions during complex thought. The breadth of this factor in humans is somewhat unique in the animal kingdoms: speech. Speech engages many parts of the brain – movement processing to coordinate the mouth and throat, listening centers to respond to a conversation, and imagination and idea processing when creating and telling stories \cite{huth}. Researchers performed MRI scans on several subjects told to read a variety of words assigned to different concepts such as “sensory,” “numeric,” or “emotional.” With a sample size of more than 10,000 words, they processed the release of neurotransmitters related to neuron activity and formed a heatmap of how concepts stimulated specific regions of the brain. It was found that while the functions are not exactly identical person-to-person, all the individuals shared many localized patterns. For example, computational concepts were clustered around one side of the prefrontal cortex (near the front of the brain), while social concepts centered on the temporal lobes (on either side of your head). This work of dividing, or “parcellating”, regions of the brain has been a centuries-old task which first began by removing parts and seeing what went wrong. This worked well for larger systems like movement or vision, but our desire to understand more complex functions necessitated more nuanced mapping. We can currently identify between 150-250 distinct regions of brain activity that are associated with things like audio processing, body sensation, or even abstract thought \cite{glasser}.

\noindent{}
\textbf{Part III – How do cells become thoughts?}

Based on the research described above, it seems that that the question of consciousness is right at the edge of our understanding. But as of now, we don’t have an exact answer. As we develop our knowledge of micro-scale interactions and brain parcellation, we can combine the two to uncover intricate connections between specific regions of the brain that may lead to the development of consciousness. There is current research in acquiring a database of neuron-based connections between the regions delineated above. These connections are tested with various stimuli to see how information is processed through the brain’s various regions, expanding on the use of artificial NNs to simulate brain activity \cite{swanson}. As breakthroughs in computing power and medical imaging continue, we will hopefully be able to create a basic simulation connecting all the regions of the brain we currently understand. Eventually, if we poke and prod at it with enough stimuli, we may get it to respond in the way we expect. 

\noindent
\textbf{Conclusion - Our knowledge of ourselves}

While we are presently far from replicating the complex functionality of the human brain, researchers across the world are becoming better at identifying specific parts and how they interact. Concepts such as emotion, abstract thinking, and sociality, once thought to be unquantifiable, have shown a clear, measurable place in the brain. Medical imaging and computer science are driving this research, with more advanced data and simulations being created every day. These works represent the first step in mapping out the more nuanced properties of the brain, from cell-to-cell interactions all the way, eventually, to consciousness.

\noindent
\textbf{References}
\begingroup
\renewcommand{\section}[2]{}% https://tex.stackexchange.com/questions/22645/hiding-the-title-of-the-bibliography
%\renewcommand{\chapter}[2]{}% for other classes
\begin{thebibliography}{6}
\bibitem{glasser} 
Glasser, M. F., Coalson, T. S., Robinson, E. C., \textit{et al} (2016). 
A multi-modal parcellation of human cerebral cortex. 
\textit{Nature}, 536(7615), 171.

\bibitem{huth}
Huth, A. G., De Heer, W. A., Griffiths, T. L., \textit{et al} (2016). 
Natural speech reveals the semantic maps that tile human cerebral cortex. 
\textit{Nature}, 532(7600), 453.

\bibitem{rubinov}
Rubinov, M. and Sporns, O. (2010). 
Complex network measures of brain connectivity: uses and interpretations. 
\textit{Neuroimage}, 52(3), 1059-1069.

\bibitem{swanson}
Swanson, L. W., Hahn, J. D., and Sporns, O. (2017). 
Organizing principles for the cerebral cortex network of commissural and association connections. 
Proceedings of the National Academy of Sciences, 114(45), E9692-E9701.

\bibitem{turk}
Turk, E., Scholtens, L. H., and van den Heuvel, M. P. (2016). 
Cortical chemoarchitecture shapes macroscale effective functional connectivity patterns in macaque cerebral cortex. 
\textit{Human brain mapping}, 37(5), 1856-1865.

\bibitem{urry}
Urry, L. A., Cain, M. L., Wasserman S. A., \textit{et al} (2017) 
\textit{Campbell Biology in Focus (AP Edition)}, 2e. London, UK: Pearson Education Inc.

\end{thebibliography}
\endgroup